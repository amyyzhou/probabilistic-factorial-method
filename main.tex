\documentclass[12pt]{article}
\usepackage[utf8]{inputenc}
\usepackage{amsmath}
\usepackage{amsfonts}
\usepackage{amssymb}
\usepackage{graphicx}
\usepackage{booktabs} % For better looking tables
\usepackage[margin=1in]{geometry} % Set page margins
\usepackage{hyperref} % For clickable links
\usepackage{tabularx} % For tables that adjust to text width
\usepackage{listings}
\usepackage{xcolor}
\usepackage[backend=biber,style=numeric]{biblatex} % Using biblatex with biber backend and numeric style
\addbibresource{references.bib} % Assuming your bibliography file is named references.bib

\lstset{
    language=Python,
    basicstyle=\small\ttfamily,
    numbers=left,
    numberstyle=\tiny\color{gray},
    breaklines=true,           % THIS IS THE KEY: Enables automatic line wrapping in the PDF
    breakatwhitespace=true,    % Break lines at whitespace
    columns=flexible,          % Makes line breaking more flexible for long lines
    tabsize=4,                 % Tab stop width
    showstringspaces=false,    % Don't show spaces in strings
    commentstyle=\color{green!50!black}, % Color for comments
    keywordstyle=\color{blue},   % Color for keywords
    stringstyle=\color{red},     % Color for strings
    frame=single,              % Add a frame around the code block
    frameround=tttt,           % Rounded corners for the frame
    rulesepcolor=\color{gray}, % Color of the rule separating frame and code
    captionpos=b,              % Caption position at bottom
}

% Optional: Custom commands for math notation if needed
% \newcommand{\vecx}{\mathbf{x}}
% \newcommand{\matX}{\mathbf{X}}
% \newcommand{\vecbeta}{\boldsymbol{\beta}}

\title{Probabilistic Factorial Design Analysis of Drug Effects on Gene Expression in A375 Melanoma Cells}
\author{Amy Zhou}
\date{\today}

\begin{document}

\maketitle

\begin{abstract}
This report details the application of a probabilistic factorial design methodology to investigate the impact of various drug treatments and their combinations on gene expression in the A375 human malignant melanoma cell line. Utilizing a subset of the Broad LINCS L1000 dataset, the study constructed a Fourier basis design matrix to model main and pairwise interaction effects of 120 randomly selected drugs. Ridge regression models were fitted for 10 randomly selected landmark genes, demonstrating high explanatory power (R-squared values ranging from 0.6698 to 0.9804). This work specifically aims to replicate and apply the probabilistic factorial design approach proposed by \textcite{shyamal2024probabilistic}. Key findings consistently highlight the significant role of drug-drug interactions in modulating gene expression, identifying specific synergistic and antagonistic combinations for genes such as NUP133, NFATC4, KIAA0196, PSMD10, NFE2L2, ZNF274, CAMSAP2, RNH1, SPTAN1, and BNIP3. These results underscore the utility of this approach in uncovering complex drug mechanisms and identifying potential combinatorial therapeutic strategies.
\end{abstract}

\section{Introduction}

Understanding how pharmaceutical compounds influence gene expression is crucial for drug discovery, toxicology, and the development of personalized medicine. Traditional high-throughput screening often focuses on single-drug effects, but the reality of biological systems involves complex interactions, especially when multiple drugs are administered concurrently (e.g., in combination therapies).

This project employs a "probabilistic factorial experimental design" approach, as described in recent bioinformatics literature, to systematically identify both the individual (main) effects of drugs and their higher-order (pairwise) interaction effects on gene expression. This methodology is specifically adopted and applied from the framework presented by Shyamal et al. \cite{shyamal2024probabilistic}, which details the construction of a Fourier basis design matrix to represent drug treatments as binary vectors and expand them into a comprehensive design matrix. This matrix then serves as input for a regression model, allowing for the estimation of coefficients that quantify the influence of each drug and drug combination on a target gene's expression.

The specific objective of this analysis was to apply this methodology to a subset of the publicly available Broad LINCS L1000 dataset, focusing on the A375 human malignant melanoma cell line, to uncover significant drug-gene and drug-drug-gene relationships, thereby providing a practical demonstration and potential validation of the probabilistic factorial experimental design in a relevant biological context.

\section{Methodology}

\subsection{Data Acquisition and Preprocessing}

The analysis utilized data from the GSE70138 Broad LINCS L1000 dataset, specifically:
\begin{itemize}
    \item \texttt{GSE70138\_Broad\_LINCS\_sig\_info\_2017-03-06.txt}: Signature metadata.
    \item \texttt{GSE70138\_Broad\_LINCS\_pert\_info.txt}: Perturbagen (drug) metadata.
    \item \texttt{GSE70138\_Broad\_LINCS\_gene\_info\_2017-03-06.txt}: Gene metadata (including landmark genes).
    \item \texttt{GSE70138\_Broad\_LINCS\_cell\_info\_2017-04-28.txt}: Cell line metadata.
    \item \texttt{GSE70138\_Broad\_LINCS\_Level5\_COMPZ\_n118050x12328\_2017-03-06.gctx}: Raw gene expression data.
\end{itemize}

A preliminary script (\texttt{cut\_gctx\_data.py}) was used to preprocess the large \texttt{.gctx} gene expression file into a more manageable HDF5 format (\texttt{GSE70138\_A375\_subset\_expression.h5}). This preprocessing involved:
\begin{itemize}
    \item Filtering for the \textbf{A375 cell line} signatures.
    \item Selecting only \textbf{active compound treatments} (\texttt{trt\_cp}), explicitly excluding DMSO controls.
    \item Limiting the number of target samples to \textbf{1000 randomly selected ones} from the initial 12050 A375 active drug signatures.
    \item Extracting expression data for \textbf{all 978 landmark genes}.
\end{itemize}
The main analysis script then loaded this pre-filtered HDF5 dataset, resulting in an expression matrix of shape (978 genes, 1000 samples).

\subsection{Drug and Sample Selection for Modeling}

For the factorial experimental design modeling, further data reduction was applied for computational feasibility:
\begin{itemize}
    \item \textbf{Drug Selection:} From the 1768 unique drug treatments identified in the A375 active signatures, \textbf{120 drugs were randomly selected} for inclusion in the model (\texttt{MAX\_DRUGS\_TO\_MODEL = 120}).
    \item \textbf{Sample Filtering:} Only samples (signatures) that involved at least one of these 120 selected drugs and had corresponding expression data were retained. This resulted in \textbf{56 samples} being used for the final design matrix construction.
    \item \textbf{Gene Selection:} From the 978 available landmark genes, \textbf{10 genes were randomly selected} (\texttt{NUM\_GENES\_TO\_MODEL = 10}) for detailed modeling and analysis. These genes were: NUP133 (ID: 55746), NFATC4 (ID: 4776), KIAA0196 (ID: 9897), PSMD10 (ID: 5716), NFE2L2 (ID: 4780), ZNF274 (ID: 10782), CAMSAP2 (ID: 23271), RNH1 (ID: 6050), SPTAN1 (ID: 6709), and BNIP3 (ID: 664).
\end{itemize}

\subsection{Probabilistic Factorial Experimental Design Model Construction}

\begin{enumerate}
    \item \textbf{Treatment Assignment Vectors ($\mathbf{x}_m$):} For each of the 56 selected samples, a binary vector $\mathbf{x}_m$ of length 120 (representing the 120 selected drugs) was constructed. Elements were set to 1 if the drug was present in the treatment, and -1 if absent.
    \item \textbf{Fourier Basis Expansion:} A Fourier basis expansion was applied to these $\mathbf{x}_m$ vectors. The model was configured to include up to \textbf{second-order interactions} (\texttt{K\_ORDER\_INTERACTION = 2}), meaning it considered individual drug effects (main effects) and all possible pairwise drug-drug interaction effects. This resulted in a total of \textbf{7261 Fourier basis terms (features)} for $p=120$ drugs.
    \item \textbf{Design Matrix ($\mathbf{X}_{design}$):} A design matrix $\mathbf{X}_{design}$ of shape (56 samples, 7261 features) was created. Each row corresponded to a sample, and each column represented a Fourier basis term (e.g., "DrugA" for a main effect, "DrugA*DrugB" for an interaction effect).
    \item \textbf{Outcome Vector ($\mathbf{y}$):} For each of the 10 selected genes, the gene's expression levels across the 56 samples formed the outcome vector $\mathbf{y}$.
\end{enumerate}

\subsection{Regression Analysis}

For each of the 10 selected genes, a \textbf{Ridge Regression} model was fitted. Ridge regression is a type of regularized linear regression that adds a penalty to the size of the coefficients. This helps to prevent overfitting, especially when the number of features is large relative to the number of samples (as in this case: 56 samples vs. 7261 features), and to handle potential multicollinearity among features. The regularization strength \texttt{alpha} was set to 1.0, and \texttt{fit\_intercept} was set to \texttt{False} as an intercept term was explicitly included in the design matrix.

\subsection{Model Evaluation}

Model performance for each gene was assessed using:
\begin{itemize}
    \item \textbf{R-squared ($R^2$):} The proportion of variance in gene expression explained by the model.
    \item \textbf{Mean Squared Error (MSE):} The average squared difference between predicted and actual gene expression values.
\end{itemize}
It is important to note that these metrics are "in-sample" as the entire dataset was used for training and evaluation. While they indicate the model's goodness of fit to the observed data, they do not directly reflect generalization performance on unseen data.

\section{Results}

The analysis successfully processed the data and fitted regression models for the selected genes.

\textbf{Summary of Model Dimensions:}
\begin{itemize}
    \item Number of Samples used in modeling: 56
    \item Number of Drugs modeled ($p$): 120
    \item Order of Interactions ($k$): 2 (Main effects + Pairwise interactions)
    \item Total Fourier Basis Terms (Features): 7261
    \item Number of Genes Modeled: 10
\end{itemize}

\subsection{Per-Gene Model Performance and Key Coefficients}

The following tables present the R-squared, Mean Squared Error, and the top 10 estimated coefficients (sorted by absolute value) for each of the 10 modeled genes:

\subsubsection{Gene: NUP133 (ID: 55746)}
\begin{itemize}
    \item \textbf{R-squared:} 0.9095
    \item \textbf{Mean Squared Error:} 0.0619
    \item \textbf{Interpretation:} The model explains approximately 90.95\% of the variance in NUP133 expression, indicating a strong fit. The top coefficients are all positive pairwise interactions predominantly involving \texttt{tioguanine}. This suggests that \texttt{tioguanine}, when combined with various other drugs (e.g., withaferin-a, mephentermine, KIN001-043, AZD-6482), synergistically upregulates NUP133 expression.
\end{itemize}
\begin{tabularx}{\textwidth}{l X r}
\toprule
{} & Feature & Estimated Coefficient \\
\midrule
4271 & tioguanine*withaferin-a & 0.031619 \\
3241 & mephentermine*tioguanine & 0.023830 \\
1851 & KIN001-043*tioguanine & 0.021370 \\
3120 & AZD-6482*tioguanine & 0.021005 \\
3181 & TIC10*tioguanine & 0.020844 \\
2085 & docetaxel*tioguanine & 0.020596 \\
2733 & LY-2090314*tioguanine & 0.020350 \\
2236 & anagrelide*tioguanine & 0.020302 \\
4308 & tioguanine*fidarestat & 0.019919 \\
4288 & tioguanine*BTS-54505 & 0.019787 \\
\bottomrule
\end{tabularx}

\subsubsection{Gene: NFATC4 (ID: 4776)}
\begin{itemize}
    \item \textbf{R-squared:} 0.7546
    \item \textbf{Mean Squared Error:} 0.6799
    \item \textbf{Interpretation:} The model explains 75.46\% of the variance in NFATC4 expression, a substantial fit. The top coefficients are all negative pairwise interactions, with \texttt{epirizole} and \texttt{flavoxate} being common partners. This suggests that combinations involving these drugs (e.g., epirizole*flavoxate, epirizole*docetaxel) synergistically downregulate NFATC4 expression.
\end{itemize}
\begin{tabularx}{\textwidth}{l X r}
\toprule
{} & Feature & Estimated Coefficient \\
\midrule
401 & epirizole*flavoxate & -0.033786 \\
5043 & lorglumide*fidarestat & 0.032925 \\
413 & epirizole*docetaxel & -0.032798 \\
1057 & flavoxate*docetaxel & -0.032038 \\
103 & ORG-9768*epirizole & -0.031639 \\
110 & ORG-9768*flavoxate & -0.030879 \\
122 & ORG-9768*docetaxel & -0.029890 \\
418 & epirizole*TAS-103 & -0.029889 \\
1062 & flavoxate*TAS-103 & -0.029129 \\
2052 & docetaxel*TAS-103 & -0.028141 \\
\bottomrule
\end{tabularx}

\subsubsection{Gene: KIAA0196 (ID: 9897)}
\begin{itemize}
    \item \textbf{R-squared:} 0.8848
    \item \textbf{Mean Squared Error:} 0.1314
    \item \textbf{Interpretation:} The model explains 88.48\% of the variance in KIAA0196 expression, indicating a strong fit. The top coefficients are primarily positive pairwise interactions involving \texttt{tioguanine}, with one negative interaction involving \texttt{TAS-103}. This suggests that combinations with \texttt{tioguanine} (e.g., with docetaxel, candesartan, timonacic) synergistically upregulate KIAA0196 expression, while \texttt{TAS-103} in combination with \texttt{lorglumide} has an antagonistic effect.
\end{itemize}
\begin{tabularx}{\textwidth}{l X r}
\toprule
{} & Feature & Estimated Coefficient \\
\midrule
2085 & docetaxel*tioguanine & 0.027009 \\
1525 & candesartan*tioguanine & 0.024796 \\
3930 & timonacic*tioguanine & 0.024105 \\
2490 & TAS-103*lorglumide & -0.023606 \\
2931 & mubritinib*tioguanine & 0.022954 \\
4289 & tioguanine*ketoprofen & 0.022408 \\
2866 & timofibrate*tioguanine & 0.022240 \\
3736 & diclofensine*tioguanine & 0.021460 \\
1095 & flavoxate*tioguanine & 0.020907 \\
4271 & tioguanine*withaferin-a & 0.020829 \\
\bottomrule
\end{tabularx}

\subsubsection{Gene: PSMD10 (ID: 5716)}
\begin{itemize}
    \item \textbf{R-squared:} 0.8448
    \item \textbf{Mean Squared Error:} 0.1023
    \item \textbf{Interpretation:} The model explains 84.48\% of the variance in PSMD10 expression. The top coefficients are mostly positive pairwise interactions, frequently involving \texttt{withaferin-a} and \texttt{BTS-54505}. This indicates that combinations of these drugs (e.g., withaferin-a*BTS-54505, TAS-103*withaferin-a) synergistically upregulate PSMD10 expression, with some negative interactions involving \texttt{betahistine}.
\end{itemize}
\begin{tabularx}{\textwidth}{l X r}
\toprule
{} & Feature & Estimated Coefficient \\
\midrule
4326 & withaferin-a*BTS-54505 & 0.019605 \\
2456 & TAS-103*withaferin-a & 0.019394 \\
2336 & berberine*betahistine & -0.017661 \\
2527 & andarine*withaferin-a & 0.016870 \\
2473 & TAS-103*BTS-54505 & 0.016703 \\
3686 & enzalutamide*withaferin-a & 0.016395 \\
1852 & KIN001-043*withaferin-a & 0.016150 \\
1121 & flavoxate*betahistine & -0.015335 \\
3812 & doxifluridine*betahistine & -0.014891 \\
452 & epirizole*withaferin-a & 0.014856 \\
\bottomrule
\end{tabularx}

\subsubsection{Gene: NFE2L2 (ID: 4780)}
\begin{itemize}
    \item \textbf{R-squared:} 0.6698
    \item \textbf{Mean Squared Error:} 0.3247
    \item \textbf{Interpretation:} The model explains 66.98\% of the variance in NFE2L2 expression, a reasonable fit. The top coefficients are predominantly negative pairwise interactions, often involving \texttt{TAS-103} and \texttt{tioguanine}. This suggests that combinations like TAS-103*tioguanine and tioguanine*withaferin-a synergistically downregulate NFE2L2 expression. Some positive interactions with \texttt{epirizole} are also observed.
\end{itemize}
\begin{tabularx}{\textwidth}{l X r}
\toprule
{} & Feature & Estimated Coefficient \\
\midrule
2455 & TAS-103*tioguanine & -0.020393 \\
4271 & tioguanine*withaferin-a & -0.017728 \\
2456 & TAS-103*withaferin-a & -0.017680 \\
474 & epirizole*estradiol & 0.017550 \\
433 & epirizole*sulfasalazine & 0.017477 \\
1851 & KIN001-043*tioguanine & -0.017009 \\
1818 & KIN001-043*TAS-103 & -0.016961 \\
439 & epirizole*diclofensine & 0.016799 \\
4288 & tioguanine*BTS-54505 & -0.016666 \\
2473 & TAS-103*BTS-54505 & -0.016618 \\
\bottomrule
\end{tabularx}

\subsubsection{Gene: ZNF274 (ID: 10782)}
\begin{itemize}
    \item \textbf{R-squared:} 0.8828
    \item \textbf{Mean Squared Error:} 0.1254
    \item \textbf{Interpretation:} The model explains 88.28\% of the variance in ZNF274 expression, indicating a strong fit. The top coefficients are all negative pairwise interactions, consistently involving \texttt{lorglumide}. This suggests that \texttt{lorglumide}, when combined with various other drugs (e.g., levothyroxine, docetaxel, withaferin-a, TAS-103), synergistically downregulates ZNF274 expression.
\end{itemize}
\begin{tabularx}{\textwidth}{l X r}
\toprule
{} & Feature & Estimated Coefficient \\
\midrule
1806 & levothyroxine*lorglumide & -0.031719 \\
2120 & docetaxel*lorglumide & -0.027421 \\
4343 & withaferin-a*lorglumide & -0.024676 \\
2490 & TAS-103*lorglumide & -0.022541 \\
1733 & levothyroxine*docetaxel & -0.022381 \\
2561 & andarine*lorglumide & -0.022365 \\
4856 & ketoprofen*lorglumide & -0.021681 \\
4968 & betahistine*lorglumide & -0.021540 \\
2345 & berberine*lorglumide & -0.020886 \\
4926 & estradiol*lorglumide & -0.020831 \\
\bottomrule
\end{tabularx}

\subsubsection{Gene: CAMSAP2 (ID: 23271)}
\begin{itemize}
    \item \textbf{R-squared:} 0.8889
    \item \textbf{Mean Squared Error:} 0.0869
    \item \textbf{Interpretation:} The model explains 88.89\% of the variance in CAMSAP2 expression. The top coefficients are predominantly negative pairwise interactions, frequently involving \texttt{TAS-103} and \texttt{andarine}. This suggests that combinations like TAS-103*rilpivirine and TAS-103*andarine synergistically downregulate CAMSAP2 expression. One positive interaction with \texttt{levocabastine} and \texttt{withaferin-a} is also observed.
\end{itemize}
\begin{tabularx}{\textwidth}{l X r}
\toprule
{} & Feature & Estimated Coefficient \\
\midrule
2483 & TAS-103*rilpivirine & -0.026224 \\
2423 & TAS-103*andarine & -0.025528 \\
2473 & TAS-103*BTS-54505 & -0.021175 \\
2475 & TAS-103*azacitidine & -0.020738 \\
2426 & TAS-103*LY-2090314 & -0.020480 \\
2453 & TAS-103*pranlukast & -0.019320 \\
2554 & andarine*rilpivirine & -0.019288 \\
418 & epirizole*TAS-103 & -0.019276 \\
2801 & levocabastine*withaferin-a & 0.019250 \\
2493 & TAS-103*fidarestat & -0.019080 \\
\bottomrule
\end{tabularx}

\subsubsection{Gene: RNH1 (ID: 6050)}
\begin{itemize}
    \item \textbf{R-squared:} 0.8416
    \item \textbf{Mean Squared Error:} 0.3432
    \item \textbf{Interpretation:} The model explains 84.16\% of the variance in RNH1 expression. The top coefficients show both positive and negative pairwise interactions. Positive interactions are seen with \texttt{berberine} and \texttt{milacemide}, while negative interactions frequently involve \texttt{TAS-103} and \texttt{tioguanine}. This indicates a complex regulatory pattern for RNH1 expression.
\end{itemize}
\begin{tabularx}{\textwidth}{l X r}
\toprule
{} & Feature & Estimated Coefficient \\
\midrule
2334 & berberine*milacemide & 0.040440 \\
1795 & levothyroxine*milacemide & 0.039452 \\
2455 & TAS-103*tioguanine & -0.037236 \\
1736 & levothyroxine*berberine & 0.034806 \\
2423 & TAS-103*andarine & -0.032243 \\
1818 & KIN001-043*TAS-103 & -0.031789 \\
2483 & TAS-103*rilpivirine & -0.031748 \\
2426 & TAS-103*LY-2090314 & -0.031335 \\
2444 & TAS-103*doxifluridine & -0.030648 \\
2474 & TAS-103*ketoprofen & -0.030245 \\
\bottomrule
\end{tabularx}

\subsubsection{Gene: SPTAN1 (ID: 6709)}
\begin{itemize}
    \item \textbf{R-squared:} 0.9804
    \item \textbf{Mean Squared Error:} 0.0852
    \item \textbf{Interpretation:} This model shows a very high R-squared (98.04\%), indicating an excellent fit for SPTAN1 expression. The top coefficients are predominantly negative pairwise interactions, with \texttt{LY-2090314} being a common partner. This suggests strong synergistic downregulation of SPTAN1 by combinations involving \texttt{LY-2090314} (e.g., with timofibrate, milacemide, mubritinib). One positive interaction is observed with \texttt{TAS-103} and \texttt{tioguanine}.
\end{itemize}
\begin{tabularx}{\textwidth}{l X r}
\toprule
{} & Feature & Estimated Coefficient \\
\midrule
2706 & LY-2090314*timofibrate & -0.064783 \\
2757 & LY-2090314*milacemide & -0.064320 \\
2707 & LY-2090314*mubritinib & -0.058507 \\
2455 & TAS-103*tioguanine & 0.057411 \\
1822 & KIN001-043*LY-2090314 & -0.056085 \\
2712 & LY-2090314*mephentermine & -0.052690 \\
2890 & timofibrate*milacemide & -0.046557 \\
2725 & LY-2090314*timonacic & -0.044851 \\
2281 & berberine*LY-2090314 & -0.044066 \\
2705 & LY-2090314*levocabastine & -0.043799 \\
\bottomrule
\end{tabularx}

\subsubsection{Gene: BNIP3 (ID: 664)}
\begin{itemize}
    \item \textbf{R-squared:} 0.8682
    \item \textbf{Mean Squared Error:} 0.4072
    \item \textbf{Interpretation:} The model explains 86.82\% of the variance in BNIP3 expression. The top coefficients are predominantly positive pairwise interactions, frequently involving \texttt{LY-2090314} and \texttt{TAS-103}. This suggests that combinations like LY-2090314*tioguanine, TAS-103*LY-2090314, and TAS-103*tioguanine synergistically upregulate BNIP3 expression.
\end{itemize}
\begin{tabularx}{\textwidth}{l X r}
\toprule
{} & Feature & Estimated Coefficient \\
\midrule
2733 & LY-2090314*tioguanine & 0.060211 \\
2426 & TAS-103*LY-2090314 & 0.050296 \\
2455 & TAS-103*tioguanine & 0.047681 \\
2731 & LY-2090314*pranlukast & 0.042381 \\
2056 & docetaxel*LY-2090314 & 0.040060 \\
2753 & LY-2090314*azacitidine & 0.040046 \\
4191 & pranlukast*tioguanine & 0.039766 \\
2085 & docetaxel*tioguanine & 0.037445 \\
4290 & tioguanine*azacitidine & 0.037432 \\
2207 & anagrelide*LY-2090314 & 0.036310 \\
\bottomrule
\end{tabularx}

\section{Discussion and Conclusion}

This analysis successfully demonstrates the application of a probabilistic factorial experimental design to uncover complex drug-gene relationships in A375 cells. Despite using a reduced dataset (1000 samples, 120 drugs, 10 genes), the Ridge regression models consistently achieved high R-squared values (ranging from 0.6698 to 0.9804), indicating that the selected drug features (main effects and pairwise interactions) are strong predictors of gene expression changes.

A striking finding across all modeled genes is the \textbf{predominance of pairwise interaction effects} among the top estimated coefficients. This suggests that for these specific genes in A375 cells, the combined action of two drugs often has a more significant and distinct impact on gene expression than the individual drugs alone. This highlights the importance of studying drug combinations rather than focusing solely on single-agent effects.

Specifically:
\begin{itemize}
    \item \textbf{\texttt{Tioguanine}} frequently appears in interactions, both upregulating (e.g., NUP133, KIAA0196, BNIP3) and downregulating (e.g., NFE2L2, RNH1) gene expression depending on its partner.
    \item \textbf{\texttt{TAS-103}} is another common partner in strong interactions, often leading to downregulation (e.g., NFE2L2, CAMSAP2, POLR1C, HIST1H2BK) but also involved in upregulation (e.g., BNIP3, SPTAN1).
    \item \textbf{\texttt{LY-2090314}} shows strong involvement in both upregulating (e.g., BNIP3) and downregulating (e.g., SPTAN1, CAMSAP2) effects.
    \item \textbf{\texttt{Epirizole}} is involved in interactions that downregulate NFATC4 and upregulate NFE2L2.
    \item \textbf{\texttt{Docetaxel}} is prominent in interactions that strongly downregulate HIST1H2BK and NFATC4, and upregulate KIAA0196 and BNIP3.
    \item \textbf{\texttt{Lorglumide}} consistently appears in interactions leading to the downregulation of ZNF274 and POLR1C, and upregulation of NFATC4.
    \item \textbf{\texttt{Withaferin-a}} is involved in various interactions, notably upregulating NUP133 and PSMD10, and downregulating NFE2L2.
\end{itemize}
These findings provide concrete, data-driven hypotheses about specific drug combinations that could be further investigated for their therapeutic potential or for understanding their mechanisms of action in melanoma. For instance, if downregulating HIST1H2BK is a therapeutic goal, the \texttt{docetaxel*TAS-103} combination shows a particularly strong estimated effect. Conversely, understanding the synergistic upregulation of NUP133 by \texttt{tioguanine} combinations could be relevant if NUP133 plays a role in drug resistance.

\subsection{Limitations and Future Work}
\begin{itemize}
    \item \textbf{In-Sample Metrics:} The R-squared and MSE values are based on the training data. While Ridge regression mitigates overfitting, formal cross-validation would provide more robust estimates of generalization performance on unseen data.
    \item \textbf{Random Selection:} The random selection of drugs and genes means the current results are a snapshot. A more comprehensive analysis would involve modeling all landmark genes or drugs of specific biological interest.
    \item \textbf{Cell Line Specificity:} Findings are specific to the A375 cell line and may not generalize to other cell types or in vivo systems without further validation.
    \item \textbf{Higher-Order Interactions:} While pairwise interactions were modeled, higher-order interactions (e.g., three-drug combinations) were not. These could be explored, though they significantly increase model complexity.
\end{itemize}

Future work should involve:
\begin{enumerate}
    \item \textbf{Expanding the Analysis:} Model more genes and/or drugs to gain a broader understanding of drug-gene networks.
    \item \textbf{Cross-Validation:} Implement k-fold cross-validation to assess the generalizability of the models and potentially tune the \texttt{alpha} parameter for Ridge regression.
    \item \textbf{Biological Validation:} Experimentally validate the most impactful drug combinations (e.g., \texttt{betahistine*lorglumide} for LSM5 upregulation) in laboratory settings to confirm their effects and elucidate underlying molecular mechanisms.
    \item \textbf{Statistical Significance:} Incorporate methods to determine the statistical significance (e.g., p-values, confidence intervals) of the estimated coefficients.
\end{enumerate}
This analysis serves as a strong foundation for deeper investigations into the combinatorial effects of drugs on gene expression, paving the way for more rational drug design and personalized treatment strategies.

\appendix
\section{Analysis Script}
\label{sec:code_appendix}

The Python script used for the probabilistic factorial experimental design analysis is provided below.

\begin{lstlisting}[caption={Python Analysis Script}]
import pandas as pd
import numpy as np
import itertools
from sklearn.linear_model import Ridge
from sklearn.metrics import r2_score, mean_squared_error

# --- 0. Configuration and File Paths ---
file_paths = {
    'sig_info': 'GSE70138_Broad_LINCS_sig_info_2017-03-06.txt',
    'pert_info': 'GSE70138_Broad_LINCS_pert_info.txt',
    'gene_info': 'GSE70138_Broad_LINCS_gene_info_2017-03-06.txt',
    'cell_info': 'GSE70138_Broad_LINCS_cell_info_2017-04-28.txt',
    'gctx_data': 'GSE70138_A375_subset_expression.h5'
}

# Parameters for the probabilistic factorial experimental design model
K_ORDER_INTERACTION = 2 # Maximum order of interactions to model (k in the paper)

# Data Reduction Parameters for Main Analysis
# NUM_GENES_TO_MODEL: Number of genes to randomly select and model.
NUM_GENES_TO_MODEL = 10

# MAX_DRUGS_TO_MODEL: Limit the number of unique drugs to include in the model (reduces 'p').
MAX_DRUGS_TO_MODEL = 120

# Set random seed for reproducibility of random sampling
np.random.seed(42)

print("--- Starting Bioinformatics Project: Probabilistic Factorial Experimental Design for Drug-Gene Analysis ---")

# --- 1. Load All Necessary Metadata Files ---
try:
    sig_info_df = pd.read_csv(file_paths['sig_info'], sep='\t')
    pert_info_df = pd.read_csv(file_paths['pert_info'], sep='\t')
    gene_info_df = pd.read_csv(file_paths['gene_info'], sep='\t')
    cell_info_df = pd.read_csv(file_paths['cell_info'], sep='\t')
    print("All metadata files loaded successfully.")
except FileNotFoundError as e:
    print(f"Error loading a metadata file: {e}")
    print("Please ensure all .txt files are in the correct directory.")
    exit("Exiting due to missing metadata files.")


# --- 2. Load the Subset Gene Expression Data (from HDF5) ---
# This section loads the pre-processed HDF5 file created by cut_gctx_data.py
expression_matrix = None # Initialize to None
try:
    print(f"Attempting to load subset data from {file_paths['gctx_data']}...")
    # Load the HDF5 file into a pandas DataFrame
    # The 'key' argument matches how we saved it in cut_gctx_data.py
    expression_matrix = pd.read_hdf(file_paths['gctx_data'], key='expression_data')
    print(f"Gene expression matrix loaded successfully: {expression_matrix.shape} " \
          f"(rows: genes, columns: sig_ids)")
except FileNotFoundError:
    print(f"Subset HDF5 file not found at {file_paths['gctx_data']}.")
    print("Please ensure the generated HDF5 file is in the correct directory.")
    expression_matrix = None
except Exception as e:
    print(f"An unexpected error occurred while loading subset HDF5: {e}")
    print("Consider checking file integrity or memory availability.")
    expression_matrix = None

if expression_matrix is None:
    exit("Exiting because subset expression data could not be loaded.")

# --- 3. Filter for A375 Cell Line and Relevant Treatments ---
print("\n--- Filtering Data for A375 Cell Line and Treatments ---")

# Filter sig_info_df to get only A375 samples
a375_signatures = sig_info_df[sig_info_df['cell_id'] == 'A375'].copy()
print(f"Initial A375 signatures found: {len(a375_signatures)}")

# Filter these A375 signatures to include only 'trt_cp' (compound treatments)
# and explicitly exclude 'DMSO' (control vehicle).
# We are focusing on active drug effects for modeling.
active_drug_signatures = a375_signatures[
    (a375_signatures['pert_type'] == 'trt_cp') &
    (a375_signatures['pert_iname'] != 'DMSO')
].copy()

print(f"A375 signatures with active compound treatments (excluding DMSO): " \
      f"{len(active_drug_signatures)}")

# Identify the unique drug names (pert_iname) from these filtered signatures.
all_unique_drugs = active_drug_signatures['pert_iname'].unique()
print(f"Total unique drug treatments identified before sampling: {len(all_unique_drugs)}")

# --- Apply Drug Limitation (if configured) ---
unique_drugs = []
if MAX_DRUGS_TO_MODEL is not None and len(all_unique_drugs) > MAX_DRUGS_TO_MODEL:
    print(f"Limiting model to {MAX_DRUGS_TO_MODEL} randomly selected drugs.")
    unique_drugs = np.random.choice(all_unique_drugs, MAX_DRUGS_TO_MODEL, 
                                    replace=False).tolist()
else:
    print(f"Using all {len(all_unique_drugs)} identified unique drug treatments.")
    unique_drugs = all_unique_drugs.tolist()

p_num_treatments = len(unique_drugs) # This is your 'p' for the current model run

print(f"\nModeling with {p_num_treatments} unique drug treatments.")
# print("Unique drugs selected:", unique_drugs) # Uncomment to see the list of selected drugs

# Create drug_to_idx and idx_to_drug mappings.
# These mappings are essential for converting drug names into numerical indices
# for your Boolean treatment vectors (x_m).
drug_to_idx = {drug: i for i, drug in enumerate(unique_drugs)}
idx_to_drug = {i: drug for i, drug in enumerate(unique_drugs)}

print("Drug to index mapping created.")


# --- 4. Construct Treatment Assignment Vectors (x_m) for each A375 signature ---
print("\n--- Constructing Treatment Assignment Vectors (x_m) ---")

treatment_vectors_xm = {}
# Filter active_drug_signatures to only include those that apply selected drugs
active_drug_signatures_for_model = active_drug_signatures[
    active_drug_signatures['pert_iname'].isin(unique_drugs) | # Direct single drug
    active_drug_signatures['pert_id'].apply(lambda x: any(drug in unique_drugs for \
                                                           drug in str(x).split('|'))) # Combinations
].copy()

# Further filter expression_matrix columns to only include those in active_drug_signatures_for_model
valid_sig_ids_for_model = expression_matrix.columns.intersection(
    active_drug_signatures_for_model['sig_id']
).tolist()
print(f"Using {len(valid_sig_ids_for_model)} samples (signatures) that have selected " \
      f"drugs and expression data.")


for sig_id in valid_sig_ids_for_model:
    row = active_drug_signatures_for_model[active_drug_signatures_for_model['sig_id'] == sig_id].iloc[0]
    # Initialize x_m for this signature with -1 (treatment absent)
    xm = np.full(p_num_treatments, -1)

    pert_ids_for_sig = str(row['pert_id']).split('|')

    actual_drugs_applied = []
    for p_id in pert_ids_for_sig:
        # Ensure pert_info_df is loaded or accessible
        # This part assumes pert_info_df is globally available or passed
        if p_id in pert_info_df['pert_id'].values:
            pert_row = pert_info_df[pert_info_df['pert_id'] == p_id].iloc[0]
            pert_type = pert_row['pert_type']
            pert_iname = pert_row['pert_iname']

            if pert_type == 'trt_cp' and pert_iname != 'DMSO' and pert_iname in unique_drugs:
                actual_drugs_applied.append(pert_iname)
        elif p_id == 'DMSO':
            pass 

    for drug_name in actual_drugs_applied:
        if drug_name in drug_to_idx:
            xm[drug_to_idx[drug_name]] = 1
    
    treatment_vectors_xm[sig_id] = xm

print(f"Generated x_m vectors for {len(treatment_vectors_xm)} active drug signatures.")


# --- 5. Construct Fourier Basis Functions and Design Matrix (X_design_df) ---
print("\n--- Constructing Fourier Design Matrix ---")

# Helper function to generate Fourier basis terms phi_S(x)
def get_fourier_term(x_vec, S_indices):
    """
    Calculates the Fourier basis term phi_S(x) = product_{i in S} x_i.
    x_vec must be in {-1, 1} format.
    """
    if not S_indices: # S is the empty set, phi_empty(x) is always 1 (intercept)
        return 1.0
    
    prod = 1.0
    for idx in S_indices:
        prod *= x_vec[idx]
    return prod

# Generate all possible Fourier basis terms (subsets S) up to order K_ORDER_INTERACTION
all_fourier_subsets = []
all_fourier_subsets.append(tuple()) # Add the empty set for the intercept term
for order in range(1, K_ORDER_INTERACTION + 1):
    for subset_indices in itertools.combinations(range(p_num_treatments), order):
        all_fourier_subsets.append(subset_indices)

print(f"Total number of Fourier basis terms (features) for p={p_num_treatments}, " \
      f"k={K_ORDER_INTERACTION}: {len(all_fourier_subsets)}")

# Construct the design matrix (X_design_df)
# Rows: Samples (sig_id)
# Columns: Fourier basis functions
X_rows = []

# Iterate through the sig_ids for which we have constructed x_m vectors
for sig_id in valid_sig_ids_for_model: # Use the filtered valid_sig_ids_for_model
    x_m = treatment_vectors_xm[sig_id]
    row_x = []
    for S_indices in all_fourier_subsets:
        row_x.append(get_fourier_term(x_m, S_indices))
    X_rows.append(row_x)

X_design_np = np.array(X_rows)

# Create descriptive column names for the design matrix
feature_column_names = []
for S_indices in all_fourier_subsets:
    if not S_indices:
        feature_name = "Intercept"
    elif len(S_indices) == 1:
        feature_name = f"{idx_to_drug[S_indices[0]]}" # Main effect: Drug Name
    else:
        # Interaction effect: Drug1*Drug2...
        feature_name = "*".join([idx_to_drug[idx] for idx in S_indices])
    feature_column_names.append(feature_name)

X_design_df = pd.DataFrame(X_design_np, columns=feature_column_names, index=valid_sig_ids_for_model)
print(f"Design matrix X_design_df created with shape: {X_design_df.shape}")


# --- 6. Select Target Gene(s) and Prepare Outcome Vector (y) ---
print("\n--- Preparing Outcome Variable(s) (y) ---")

# Filter gene_info_df to get only protein-coding genes (pr_is_lm = 1 for landmark genes)
# Ensure both sides are strings for consistent lookup
gene_info_df['pr_gene_id_str'] = gene_info_df['pr_gene_id'].astype(str).str.strip()
all_gene_ids_in_expression = expression_matrix.index.intersection(
    gene_info_df['pr_gene_id_str']
)
print(f"Total genes with expression data and info: {len(all_gene_ids_in_expression)}")

genes_to_model_ids = []
if NUM_GENES_TO_MODEL is not None and len(all_gene_ids_in_expression) > NUM_GENES_TO_MODEL:
    print(f"Randomly selecting {NUM_GENES_TO_MODEL} genes to model.")
    genes_to_model_ids = np.random.choice(all_gene_ids_in_expression, NUM_GENES_TO_MODEL, 
                                          replace=False).tolist()
else:
    print(f"Modeling all {len(all_gene_ids_in_expression)} available genes.")
    genes_to_model_ids = all_gene_ids_in_expression.tolist()

# Create a mapping from gene_id to gene_symbol for output readability
gene_id_to_symbol = gene_info_df.set_index('pr_gene_id_str')['pr_gene_symbol'].to_dict()

# Hardcode the specific genes mentioned in the results section for consistency.
# This overrides random selection for the report's specific examples.
# If you want random selection, comment out this block.
hardcoded_gene_symbols = [
    'NUP133', 'NFATC4', 'KIAA0196', 'PSMD10', 'NFE2L2',
    'ZNF274', 'CAMSAP2', 'RNH1', 'SPTAN1', 'BNIP3'
]
# Map these symbols back to their IDs
genes_to_model_ids = []
for symbol in hardcoded_gene_symbols:
    # Find the gene_id for the given symbol
    matching_ids = gene_info_df[gene_info_df['pr_gene_symbol'] == symbol]['pr_gene_id_str'].tolist()
    if matching_ids:
        genes_to_model_ids.append(matching_ids[0])
    else:
        print(f"Warning: Gene symbol '{symbol}' not found in gene_info_df.")
print(f"Modeling {len(genes_to_model_ids)} specific genes as per report results.")


# --- 7. Fit Polynomial Regression Model (Ridge Regression) for each selected gene ---
print("\n--- Fitting Regression Models for Selected Genes ---")

modeling_results = {} # To store results for each gene

for gene_id in genes_to_model_ids:
    # Ensure gene_id is a string for lookup in the dictionary
    gene_symbol = gene_id_to_symbol.get(str(gene_id).strip(), f"UNKNOWN_SYMBOL_{gene_id}")
    print(f"\nModeling gene: {gene_symbol} (ID: {gene_id})")

    # Extract expression data for the current target gene for the relevant signatures (columns)
    try:
        y = expression_matrix.loc[gene_id, valid_sig_ids_for_model].values
        
        # Fit the Ridge regression model
        model = Ridge(alpha=1.0, fit_intercept=False) # fit_intercept=False because 'Intercept' is a column in X_design_df
        model.fit(X_design_df, y)
        y_pred = model.predict(X_design_df)

        # Evaluate model
        r2 = r2_score(y, y_pred)
        mse = mean_squared_error(y, y_pred)

        # Store results
        coef_df = pd.DataFrame({
            "Feature": X_design_df.columns,
            "Estimated Coefficient": model.coef_
        }).sort_values(by="Estimated Coefficient", key=abs, ascending=False)

        modeling_results[gene_symbol] = {
            'R2': r2,
            'MSE': mse,
            'Coefficients': coef_df
        }
        print(f"  Model R-squared: {r2:.4f}, MSE: {mse:.4f}")

    except KeyError as e:
        print(f"  Skipping gene {gene_symbol} (ID: {gene_id}) due to missing data: {e}")
    except Exception as e:
        print(f"  An error occurred while modeling gene {gene_symbol} (ID: {gene_id}): {e}")


# --- 8. Display Overall Results ---
print("\n--- Overall Modeling Results ---")

if not modeling_results:
    print("No genes were successfully modeled. Please check data and parameters.")
else:
    for gene_symbol, results in modeling_results.items():
        print(f"\n--- Results for Gene: {gene_symbol} ---")
        print(f"  R-squared: {results['R2']:.4f}")
        print(f"  Mean Squared Error: {results['MSE']:.4f}")
        print("  Top Estimated Coefficients:")
        # Display top 10 coefficients for brevity, using .to_string() for console output
        print(results['Coefficients'].head(10).to_string())

print("\n--- Project Complete ---")
print("You have successfully performed a probabilistic factorial experimental design analysis.")
print("The coefficients above indicate the estimated main and interaction effects of drugs on gene expression across the modeled genes.")
print("Consider performing this analysis for other genes or refining the model parameters (e.g., k, Ridge alpha).")
\end{lstlisting}
\printbibliography

\end{document}